% Copyright 2004 by Till Tantau <tantau@users.sourceforge.net>.
%
% In principle, this file can be redistributed and/or modified under
% the terms of the GNU Public License, version 2.
%
% However, this file is supposed to be a template to be modified
% for your own needs. For this reason, if you use this file as a
% template and not specifically distribute it as part of a another
% package/program, I grant the extra permission to freely copy and
% modify this file as you see fit and even to delete this copyright
% notice. 


\documentclass[19pt]{beamer}

% There are many different themes available for Beamer. A comprehensive
% list with examples is given here:
% http://deic.uab.es/~iblanes/beamer_gallery/index_by_theme.html
% You can uncomment the themes below if you would like to use a different
% one:
%\usetheme{AnnArbor}
%\usetheme{Antibes}
%\usetheme{Bergen}
%\usetheme{Berkeley}
%\usetheme{Berlin}
%\usetheme{Boadilla}
%\usetheme{boxes}
%\usetheme{CambridgeUS}
%\usetheme{Copenhagen}
%\usetheme{Darmstadt}
%\usetheme{default}
%\usetheme{Frankfurt}
%\usetheme{Goettingen}
%\usetheme{Hannover}
%\usetheme{Ilmenau}
%\usetheme{JuanLesPins}
%\usetheme{Luebeck}
\usetheme{Madrid}
%\usetheme{Malmoe}
%\usetheme{Marburg}
%\usetheme{Montpellier}
%\usetheme{PaloAlto}
%\usetheme{Pittsburgh}
%\usetheme{Rochester}
%\usetheme{Singapore}
%\usetheme{Szeged}
%\usetheme{Warsaw}

\usepackage[T2A,T1]{fontenc}
\usepackage[utf8]{inputenc}
\usepackage[russian]{babel}
\usepackage[normalem]{ulem}
\usepackage{multicol}
\usepackage{graphicx}
\graphicspath{ {images/} }
\usepackage{caption}
\usepackage{animate}
\usepackage{multimedia}

\title{Deep Weight Prior}

% A subtitle is optional and this may be deleted

\author{Kotova Darya}
% - Give the names in the same order as the appear in the paper.
% - Use the \inst{?} command only if the authors have different
%   affiliation.

\institute[MIPT] % (optional, but mostly needed)

% - Use the \inst command only if there are several affiliations.
% - Keep it simple, no one is interested in your street address.

\date{May, 2019}
% - Either use conference name or its abbreviation.
% - Not really informative to the audience, more for people (including
%   yourself) who are reading the slides online

\subject{}
% This is only inserted into the PDF information catalog. Can be left
% out. 

% If you have a file called "university-logo-filename.xxx", where xxx
% is a graphic format that can be processed by latex or pdflatex,
% resp., then you can add a logo as follows:

% \pgfdeclareimage[height=0.5cm]{university-logo}{university-logo-filename}
% \logo{\pgfuseimage{university-logo}}

% Delete this, if you do not want the table of contents to pop up at
% the beginning of each subsection:
% \AtBeginSubsection[]
% {
%   \begin{frame}<beamer>{План}
%     \tableofcontents[currentsection,currentsubsection]
%   \end{frame}
% }

% Let's get started
\begin{document}

\begin{frame}
  \titlepage
\end{frame}

\begin{frame}{Contents}
  \tableofcontents
  % You might wish to add the option [pausesections]
\end{frame}

% Section and subsections will appear in the presentation overview
% and table of contents.
\section{Variational inference}

%3
\begin{frame}{Variational Inference: Problem}
Goal: transform prior knowledge $p(w)$ to the posterior distribution $p(w \vert D)$ with Bayes rule:\\
$$
p(w \vert D) = \frac{p(D \vert w) p(w)}{p(D)}
$$\\
The problem is $p(D)$ is unknown $\rightarrow$ $p(w \vert D)$ is intractable. \\
Then let's approximate $p(w \vert D)$ with $q_{\theta}(w)$.
\end{frame}

%4
\begin{frame}{Variational lower bound}

To make it we minimize KL-divergence:\\
$$
D_{KL}(q_{\theta}(w) \vert \vert p(w \vert D)) = \mathbb{E}_{q_{\theta}}log\frac{q_{\theta}(w)}{p(w \vert  D)} \rightarrow \min_{\theta}
$$ 
It is equivalent to \textbf{variational lower bound maximization}
$$
VLB := \mathbb{E}_{q_{\theta}} \log p(D \vert w) - D_{KL}(q_{\theta}(w) \vert \vert p(w)) \rightarrow \max_{\theta}
$$
\end{frame}

\section{Reparametrisation trick}

%5
\begin{frame}{Reparametrisation trick}
$$
VLB = \mathbb{E}_{q_{\theta}} \log p(D \vert w) - D_{KL}(q_{\theta}(w) \vert \vert p(w)) \rightarrow \max_{\theta}
$$

$$ 
\nabla_{\theta} \mathbb{E}_{q_{\theta}} \log p(D| w) = \int \nabla_{\theta} q_{\theta}(w) \log p(D \vert w)  \neq \mathbb{E}_{q_{\theta}} \nabla_{\theta} \log p(D \vert w)
$$

So we can not obtain unbiased gradients and perform mini-batching training. \\

\end{frame}

%6
\begin{frame}{Reparametrisation trick}

\textbf{Idea}:\\

Let's represent $q_{\theta}$ as deterministic differentiable function $f(\theta, \epsilon)$.\\
For example:\\
$$
\epsilon \sim \mathcal{N}(0, 1) \quad
\xi \sim \mathcal{N}(\mu, \sigma^{2})
$$
$$
\epsilon = \frac{\xi - \mu}{\sigma} \quad \xi = \mu + \sigma \cdot \epsilon 
$$
\\
Now we can compute gradients of VLB, where instead of $q_{\theta}$ now is $f(\theta, \epsilon)$.

\end{frame}

%7
%yes, once again intentionally
\begin{frame}{Variational lower bound}

To make it we minimize KL-divergence:\\
$$
D_{KL}(q_{\theta}(w) \vert \vert p(w \vert D)) = \mathbb{E}_{q_{\theta}}log\frac{q_{\theta}(w)}{p(w \vert  D)} \rightarrow \min_{\theta}
$$ 
It is equivalent to \textbf{variational lower bound maximization}
$$
VLB = \mathbb{E}_{q_{\theta}} \log p(D \vert w) - D_{KL}(q_{\theta}(w) \vert \vert p(w)) \rightarrow \max_{\theta}
$$
\end{frame}

%8
\begin{frame}{Prior distribution}

We suggest that networks learned on the similar datasets will have similar kernel's structures in convolutional layers.\\

\begin{figure}
    \centering
    \includegraphics[width = 0.8\linewidth]{kernels.png}
    % \caption{Caption}
    \label{fig:my_label}
\end{figure}

\end{frame}

\section{Variational autoencoder}

%9
\begin{frame}{Variational autoencoder}

\begin{figure}
    \centering
    \includegraphics[width = 0.85\linewidth]{vae.png}
    % \caption{Caption}
    \label{fig:my_label}
\end{figure}

It helps us to approximate $p(w)$, having weights from already learned networks.

\end{frame}

%10
\begin{frame}{Variational inference with implicit prior distribution}

However, we still do not have probability density function of p(w). \\
VAE also helps to make auxilary bound for VLB:\\

$$
VLB = \mathbb{E}_{q_{\theta}} \log p(D \vert w) - D_{KL}(q_{\theta}(w) \vert \vert p(w)) = 
$$
$$
\mathbb{E}_{q(w)} \mathbb{E}_{r(z \vert w)} (log\frac{p(w, z)}{r(z \vert w)}p(x \vert w) - log(q(w)) + \mathbb{E}_{q(w)}D_{KL}(r(z \vert w) \vert \vert p(z \vert w)) =
$$
$$
= L^{aux} +  \mathbb{E}_{q(w)}D_{KL}(r(z \vert w) \vert \vert p(z \vert w)) \geq L^{aux}
$$

\end{frame}

\section{Results}

%11
\begin{frame}{Results}

Code and different examples are available on \url{https://github.com/DahaKot/Deep-Weight-Prior} \\

\begin{itemize}
    \item trained 100 CNNs on notMNIST
    \item trained vae on source kernels
    \item used samples from vae for CNN learned on MNIST
    \item compare performance of CNNs with different initialization
\end{itemize}

\end{frame}

%12
\begin{frame}{Kernel examples}

There are samples of the source kernels:\\

\begin{figure}
    \centering
    \begin{subfigure}%[b]{0.1\textwidth}
        \includegraphics[width=0.15\textwidth]{sourcekernel1.jpg}
        %\caption{A gull}
        %\label{fig:gull}
    \end{subfigure}
    ~ %add desired spacing between images, e. g. ~, \quad, \qquad, \hfill etc. 
      %(or a blank line to force the subfigure onto a new line)
    \begin{subfigure}%[b]{0.1\textwidth}
        \includegraphics[width=0.15\textwidth]{sourcekernel2.jpg}
        % \caption{A tiger}
        \label{fig:tiger}
    \end{subfigure}
    ~ %add desired spacing between images, e. g. ~, \quad, \qquad, \hfill etc. 
    %(or a blank line to force the subfigure onto a new line)
    \begin{subfigure}%[b]{0.1\textwidth}
        \includegraphics[width=0.15\textwidth]{sourcekernel3.jpg}
        % \caption{A mouse}
        \label{fig:mouse}
    \end{subfigure}
     ~ %add desired spacing between images, e. g. ~, \quad, \qquad, \hfill etc. 
    %(or a blank line to force the subfigure onto a new line)
    \begin{subfigure}%[b]{0.1\textwidth}
        \includegraphics[width=0.15\textwidth]{sourcekernel4.jpg}
        % \caption{A mouse}
        \label{fig:mouse}
    \end{subfigure}
    %\caption{Source kernels}
    \label{fig:animals}
\end{figure}

And kernels got from vae:\\

\begin{figure}
    \centering
    \begin{subfigure}%[b]{0.1\textwidth}
        \includegraphics[width=0.15\textwidth]{vaekernel1.png}
        %\caption{A gull}
        %\label{fig:gull}
    \end{subfigure}
    ~ %add desired spacing between images, e. g. ~, \quad, \qquad, \hfill etc. 
      %(or a blank line to force the subfigure onto a new line)
    \begin{subfigure}%[b]{0.1\textwidth}
        \includegraphics[width=0.15\textwidth]{vaekernel2.png}
        % \caption{A tiger}
        \label{fig:tiger}
    \end{subfigure}
    ~ %add desired spacing between images, e. g. ~, \quad, \qquad, \hfill etc. 
    %(or a blank line to force the subfigure onto a new line)
    \begin{subfigure}%[b]{0.1\textwidth}
        \includegraphics[width=0.15\textwidth]{vaekernel3.png}
        % \caption{A mouse}
        \label{fig:mouse}
    \end{subfigure}
     ~ %add desired spacing between images, e. g. ~, \quad, \qquad, \hfill etc. 
    %(or a blank line to force the subfigure onto a new line)
    \begin{subfigure}%[b]{0.1\textwidth}
        \includegraphics[width=0.15\textwidth]{vaekernel4.png}
        % \caption{A mouse}
        \label{fig:mouse}
    \end{subfigure}
    %\caption{Source kernels}
    \label{fig:animals}
\end{figure}

\end{frame}

%13
\begin{frame}{Latent space}

Visualization of latent representations of convolutional filters
for ConvNet on notMNIST:\\

\begin{figure}
    \centering
    \includegraphics[width = 0.65\linewidth]{latentspace.png}
    % \caption{Caption}
    \label{fig:my_label}
\end{figure}

\end{frame}

%14
\begin{frame}{Performance comparation}

The performance of convolutional network with two different priors: deep weight prior (dwp) and standard normal:\\

\begin{figure}
    \centering
    \includegraphics[width = 0.7\linewidth]{performance.png}
    % \caption{Caption}
    \label{fig:my_label}
\end{figure}

\end{frame}

\section{Conclusion}

%15
\begin{frame}{Conclusion}

We considered a new way to initilize kernels of convolutional neural networks - deep weight prior:\\
\begin{itemize}
    \item Performs better than 'default' random initialization
    \item Does not need as much memory and computations as initialization with already learned filters
\end{itemize}

\end{frame}
 
%16
\begin{frame}{Thank you for attention}
 
\begin{figure}
    \centering
    \includegraphics[width = 0.45\linewidth]{mem.png}
    % \caption{Caption}
    \label{fig:my_label}
\end{figure}

\end{frame}

%17
\begin{frame}{(not sure which meme to choose)}
 
\begin{figure}
    \centering
    \includegraphics[width = 0.7\linewidth]{meme.jpg}
    % \caption{Caption}
    \label{fig:my_label}
\end{figure}

\end{frame}

\end{document}


